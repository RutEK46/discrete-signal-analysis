\chapter{Sygnały cyfrowe}
\section{Wprowadzenie}
W tym rozdziale trochę teorii - czym są sygnały cyfrowe (w ogólności - dyskretne). 


\section{Narzędzia}
Poniżej zostały zaprezentowane podstawowe narzędzia do analizy sygnałów cyfrowych.
\subsection{Średnia}
Prosta średnia ruchoma:
\begin{equation}
SMA = \frac{1}{n}\sum_{i = 0}^{n - 1}p_i,
\end{equation}
gdzie: $n$ to liczba okresów, a $p_i$ to wartość z sprzed $i$ okresów
Wykładnicza średnia ruchoma:
\begin{equation}
EMA = \sum_{i = 0}^{n - 1}((1 - \alpha)^ip_i) / \sum_{i = 0}^{n-1}(1 - \alpha)^i,
\end{equation}
gdzie: $n$ to liczba okresów, $\alpha = \frac{2}{n}$, a $p_i$ to wartość z sprzed $i$ okresów
można dać przykład: średnia krocząca w MACD
\subsection{Odchylenie standardowe}
\begin{equation}
\sigma^2 = \frac{1}{n}\sum_{i = 0}^{n - 1}(p_i - SMA)^2
\end{equation}
\subsection{Całkowanie}
todo
\subsection{Pochodne}
Najczęściej stosuje się wzór na pochodną środkową:
\begin{equation}
f^\prime(x) = \frac{f(x + h) - f(x - h)}{2h}
\end{equation}
W przypadku pochodnej na początku sygnału stosuje się pochodną prawostronną:
\begin{equation}
f^\prime(x) = \frac{f(x + h) - f(x)}{h}
\end{equation}
natomiast w przypadku pochodnej na końcu sygnału stosuje się pochodną lewostronną:
\begin{equation}
f^\prime(x) = \frac{f(x) - f(x - h)}{h}
\end{equation}
można dać przykład: pochodne w badaniu trendu, predykcji


