\chapter{Sygna�y cyfrowe}
\section{Wprowadzenie}
W tym rozdziale troch� teorii - czym s� sygna�y cyfrowe (w og�lno�ci - dyskretne). 


\section{Narz�dzia}
Poni�ej zosta�y zaprezentowane podstawowe narz�dzia do analizy sygna��w cyfrowych.
\subsection{�rednia}
Prosta �rednia ruchoma:
\begin{equation}
SMA = \frac{1}{n}\sum_{i = 0}^{n - 1}p_i,
\end{equation}
gdzie: $n$ to liczba okres�w, a $p_i$ to warto�� z sprzed $i$ okres�w
Wyk�adnicza �rednia ruchoma:
\begin{equation}
SMA = \sum_{i = 0}^{n - 1}((1 - \alpha)^ip_i) / \sum_{i = 0}^{n-1}(1 - \alpha)^i,
\end{equation}
gdzie: $n$ to liczba okres�w, $\alpha = \frac{2}{n}$, a $p_i$ to warto�� z sprzed $i$ okres�w
mo�na da� przyk�ad: �rednia krocz�ca w MACD
\subsection{Odchylenie standardowe}
\begin{equation}
\sigma^2 = \frac{1}{n}\sum_{i = 0}^{n - 1}(p_i - SMA)^2
\end{equation}
\subsection{Ca�kowanie}
todo
\subsection{Pochodne}
Najcz�ciej stosuje si� wz�r na pochodn� �rodkow�:
\begin{equation}
f^\prime(x) = \frac{f(x + h) - f(x - h)}{2h}
\end{equation}
W przypadku pochodnej na pocz�tku sygna�u stosuje si� pochodn� prawostronn�:
\begin{equation}
f^\prime(x) = \frac{f(x + h) - f(x)}{h}
\end{equation}
natomiast w przypadku pochodnej na ko�cu sygna�u stosuje si� pochodn� lewostronn�:
\begin{equation}
f^\prime(x) = \frac{f(x) - f(x - h)}{h}
\end{equation}
mo�na da� przyk�ad: pochodne w badaniu trendu, predykcji


